\documentclass[12pt,a4paper]{report}

% Page setup
\usepackage[top=2cm, bottom=2cm, left=2cm, right=2cm]{geometry}
\usepackage{setspace}
\onehalfspacing

% Font
\usepackage{times}
\usepackage[T1]{fontenc}

% Packages
\usepackage{graphicx}
\usepackage{amsmath}
\usepackage{amssymb}
\usepackage{booktabs}
\usepackage{cite}
\usepackage{url}
\usepackage{hyperref}
\usepackage{fancyhdr}
\usepackage{titlesec}

% Chapter, section formatting
\titleformat{\chapter}[display]
  {\normalfont\fontsize{18}{20}\bfseries\centering}{\chaptertitlename\ \thechapter}{0pt}{\fontsize{18}{20}\bfseries}
\titleformat{\section}
  {\normalfont\fontsize{16}{18}\bfseries}{\thesection}{1em}{}
\titleformat{\subsection}
  {\normalfont\fontsize{14}{16}\bfseries}{\thesubsection}{1em}{}

% Header and footer
\pagestyle{fancy}
\fancyhf{}
\fancyfoot[C]{\thepage}
\renewcommand{\headrulewidth}{0pt}

% Chapter opening pages
\fancypagestyle{plain}{
  \fancyhf{}
  \fancyfoot[C]{\thepage}
  \renewcommand{\headrulewidth}{0pt}
}

\begin{document}

% Inner Title Page
\begin{titlepage}
\centering
\vspace*{1cm}

{\Large\bfseries PES UNIVERSITY}\\[0.3cm]
{\normalsize (Established under Karnataka Act No. 16 of 2013)}\\[0.5cm]
{\normalsize 100 feet Ring Road, BSK 3rd stage, Hosakerehalli, Bengaluru - 560085}\\[1cm]

{\large\bfseries Department of Computer Science \& Engineering}\\[2cm]

{\LARGE\bfseries UE23CS320A - CAPSTONE PROJECT}\\[0.5cm]
{\LARGE\bfseries PHASE - 1 REPORT}\\[2cm]

{\Large\bfseries Causal Model for ECG Analysis on Myocardial Infarction}\\[3cm]

{\large Submitted by:}\\[0.5cm]
{\large Your Name}\\
{\normalsize SRN: Your SRN}\\[2cm]

{\large Under the guidance of:}\\[0.5cm]
{\large Prof. Priya Badrinath}\\
{\normalsize Department of Computer Science \& Engineering}\\[2cm]

{\large November 2025}

\end{titlepage}

% Certificate
\chapter*{Certificate}
\addcontentsline{toc}{chapter}{Certificate}
\thispagestyle{plain}

This is to certify that the project work entitled \textbf{"Causal Model for ECG Analysis on Myocardial Infarction"} submitted in partial fulfillment of the requirements for the award of degree of Bachelor of Technology in Computer Science \& Engineering at PES University, Bengaluru is a bonafide work carried out by [Your Name], SRN: [Your SRN] during the academic year 2025-26 under my guidance.

\vspace{2cm}

\noindent Prof. Priya Badrinath\\
Project Guide\\
Department of Computer Science \& Engineering\\
PES University

% Acknowledgement
\chapter*{Acknowledgement}
\addcontentsline{toc}{chapter}{Acknowledgement}
\thispagestyle{plain}

I would like to express my sincere gratitude to Prof. Priya Badrinath for her invaluable guidance and support throughout Phase-1 of this capstone project. Her expertise in machine learning and healthcare applications has been instrumental in shaping the direction of this research.

I am thankful to the Department of Computer Science \& Engineering, PES University, for providing the necessary infrastructure and resources to conduct this research. I also acknowledge the open-source datasets from MIMIC-IV and PTB-XL that made this work possible.

% Abstract
\chapter*{Abstract}
\addcontentsline{toc}{chapter}{Abstract}
\thispagestyle{plain}

Myocardial infarction (MI) is a leading cause of global mortality, yet current clinical practice lacks personalized tools to predict individual patient outcomes and treatment responses. This project aims to address this gap by developing a causal inference framework for electrocardiogram (ECG) analysis that goes beyond traditional prediction to answer questions specific to a patient, such as "What is the actual benefit of starting a statin for this particular patient?"Phase-1 of this work is focused on three key components: (1) cohort definition and data preparation from the MIMIC-IV database, (2) development and training of a $\beta$-Variational Autoencoder ($\beta$-VAE) for unsupervised feature learning from 12-lead ECGs, and (3) a comprehensive literature review for establishing the theoretical foundation for causal analysis in healthcare.We successfully labeled 47,852 ECG records from MIMIC-IV using a clinical adjudication framework based on troponin, identifying 5,958 acute MI presentations and 41,894 symptomatic controls. A Convolutional 1D $\beta$-VAE with 64 latent dimensions was trained to learn disentangled representations of ECG signals, achieving excellent reconstruction quality (validation loss: 32,722.57) and minimal overfitting (0.74% generalization gap). Preliminary analysis identified 6 discriminative latent features with strong statistical independence (mean correlation: 0.046), validating the model is able to capture meaningful physiological patterns.The literature survey spanning 15 seminal papers shows that existing MI detection methods based on ECG are predominantly associational, lacking causal interpretability. This foundational work lays the groundwork for implementing Structural Causal Models (SCM), Conditional Average Treatment Effect (CATE) estimation, and counterfactual generation at the patient level which will enable personalized treatment recommendations.

\textbf{Keywords:}Causal Inference, Myocardial Infarction, Electrocardiogram, Variational Autoencoder, Disentangled Representation Learning, MIMIC-IV

% Table of Contents
\tableofcontents
\thispagestyle{plain}

% List of Tables
\listoftables
\thispagestyle{plain}

% List of Figures
\listoffigures
\thispagestyle{plain}

% Start page numbering for main content
\setcounter{page}{1}
\pagenumbering{arabic}

%===========================================
% CHAPTER 1: INTRODUCTION
%===========================================
\chapter{Introduction}

\section{Background}

Cardiovascular diseases remain the leading cause of mortality worldwide, with myocardial infarction (MI) accounting for approximately 17.9 million deaths annually according to the World Health Organization. The electrocardiogram (ECG) is the primary diagnostic tool for MI detection, offering a non-invasive, cost-effective method for cardiac assessment. However, current clinical practice relies heavily on associational patterns—identifying correlations between ECG features and MI outcomes—without establishing causal relationships.

Machine learning models for ECG analysis have achieved remarkable predictive accuracy, with deep learning approaches reaching sensitivities exceeding 90\% for MI detection. Yet these models fundamentally answer the question "What will happen?" rather than "Why will it happen?" or "What should we do about it?" This limitation prevents clinicians from making truly personalized treatment decisions.

\section{Motivation}

Consider a 62-year-old patient presenting with chest pain and ST-segment elevation on ECG. Current risk models predict a 78\% probability of MI, but they cannot answer critical clinical questions:

\begin{itemize}
    \item If we administer aspirin immediately, how much will this patient's risk decrease?
    \item Would starting a statin provide meaningful benefit for this specific individual?
    \item Which intervention—PCI, thrombolysis, or medical management—offers the greatest survival benefit?
\end{itemize}

These questions require causal reasoning, not mere prediction. Causal inference frameworks, particularly those employing Structural Causal Models (SCM) and Conditional Average Treatment Effect (CATE) estimation, can bridge this gap by quantifying individualized treatment effects.

\section{Problem Statement}

\textbf{Primary Objective:} Develop a causal inference framework for ECG-based analysis that estimates patient-specific treatment effects for myocardial infarction management.

\textbf{Research Question:} Can we move beyond associational ECG interpretation to establish causal relationships between patient characteristics, ECG patterns, treatments, and MI outcomes?

\section{Scope of This Report}

This report documents the completion of foundational components necessary for causal analysis:

\begin{enumerate}
    \item \textbf{Data Acquisition and Cohort Definition:} Identification and labeling of MI and control cases from MIMIC-IV using clinical troponin levels and expert adjudication rules.
    
    \item \textbf{Unsupervised Feature Learning:} Development of a $\beta$-Variational Autoencoder to extract disentangled, interpretable representations from raw 12-lead ECG signals.
    
    \item \textbf{Literature Foundation:} Comprehensive survey of existing work in ECG analysis, causal inference in healthcare, and representation learning to identify research gaps.
    
    \item \textbf{Preliminary Validation:} Statistical analysis of learned latent features to ensure they capture clinically meaningful patterns and satisfy disentanglement criteria.
\end{enumerate}

This report does \textbf{not} include causal model implementation, treatment effect estimation, or full clinical validation—these components represent future work.

\section{Organization of Report}

The remainder of this report is structured as follows:

\begin{itemize}
    \item \textbf{Chapter 2} formally defines the problem, including clinical context, technical challenges, and the proposed causal framework.
    \item \textbf{Chapter 3} presents a comprehensive literature survey covering ECG-based MI detection, causal inference methods, and representation learning techniques.
    \item \textbf{Chapter 4} identifies critical research gaps and technical challenges that justify this work.
    \item \textbf{Chapter 5} details the data exploration process, including cohort statistics, feature analysis, and model training results.
\end{itemize}

%===========================================
% CHAPTER 2: PROBLEM DEFINITION
%===========================================
\chapter{Problem Definition}

\section{Clinical Context}

Myocardial infarction occurs when coronary artery occlusion leads to myocardial cell death, manifesting in characteristic ECG patterns such as ST-segment elevation, T-wave inversion, and pathological Q-waves. The American College of Cardiology defines MI diagnosis through a combination of:

\begin{enumerate}
    \item Detection of rise and/or fall in cardiac troponin levels with at least one value above the 99th percentile upper reference limit
    \item At least one of the following: symptoms of myocardial ischemia, new ischemic ECG changes, development of pathological Q waves, imaging evidence of new loss of viable myocardium
\end{enumerate}

Clinical decision-making involves multiple interventions—antiplatelet therapy (aspirin, clopidogrel), statins, percutaneous coronary intervention (PCI), and thrombolytic therapy—each with varying efficacy across patient subgroups.

\section{Technical Problem Formulation}

\subsection{Causal Inference Framework}

We adopt the potential outcomes framework (Rubin-Neyman model) where each patient $i$ has two potential outcomes:

\begin{equation}
Y_i(1) = \text{outcome if patient } i \text{ receives treatment}
\end{equation}

\begin{equation}
Y_i(0) = \text{outcome if patient } i \text{ receives control}
\end{equation}

The individual treatment effect (ITE) is defined as:

\begin{equation}
\tau_i = Y_i(1) - Y_i(0)
\end{equation}

Since we only observe one potential outcome per patient (fundamental problem of causal inference), we estimate the Conditional Average Treatment Effect (CATE):

\begin{equation}
\tau(X) = \mathbb{E}[Y(1) - Y(0) | X = x]
\end{equation}

where $X$ represents patient covariates derived from ECG features, demographics, and clinical history.

\subsection{ECG Representation Learning}

Raw 12-lead ECG signals are high-dimensional (5000 samples/lead $\times$ 12 leads = 60,000 dimensions) and contain redundant information. We formulate representation learning as:

\begin{equation}
\text{ECG Signal } \mathbf{s} \in \mathbb{R}^{60000} \rightarrow \text{Latent Embedding } \mathbf{z} \in \mathbb{R}^{64}
\end{equation}

The $\beta$-VAE learns a mapping $q_\phi(\mathbf{z}|\mathbf{s})$ (encoder) and $p_\theta(\mathbf{s}|\mathbf{z})$ (decoder) by optimizing:

\begin{equation}
\mathcal{L}(\theta, \phi) = -\mathbb{E}_{q_\phi(\mathbf{z}|\mathbf{s})}[\log p_\theta(\mathbf{s}|\mathbf{z})] + \beta \cdot D_{KL}(q_\phi(\mathbf{z}|\mathbf{s}) \| p(\mathbf{z}))
\end{equation}

where $\beta > 1$ enforces disentanglement, encouraging latent dimensions to capture independent generative factors.

\section{Key Challenges}

\subsection{Challenge 1: Class Imbalance}

MIMIC-IV exhibits severe class imbalance with MI prevalence of approximately 12.4\% (5,958 MI cases vs. 41,894 controls). This imbalance can bias causal effect estimates if not properly addressed through stratification or weighting.

\subsection{Challenge 2: Confounding Variables}

Patient age, sex, comorbidities (diabetes, hypertension), and prior cardiac history confound the relationship between ECG patterns and MI outcomes. Adjusting for these confounders is essential for unbiased CATE estimation.

\subsection{Challenge 3: Feature Interpretability}

Deep learning models often produce "black box" features that lack clinical interpretability. Disentangled representations are necessary to ensure learned features correspond to physiological concepts (e.g., heart rate variability, QRS duration, ST-segment deviation).

\subsection{Challenge 4: Causal Assumption Verification}

Estimating causal effects requires untestable assumptions:
\begin{itemize}
    \item \textbf{Ignorability:} Treatment assignment is independent of potential outcomes conditional on covariates
    \item \textbf{Positivity:} All patients have non-zero probability of receiving any treatment
    \item \textbf{Stable Unit Treatment Value Assumption (SUTVA):} One patient's treatment does not affect another's outcome
\end{itemize}

While we cannot verify these assumptions directly, sensitivity analysis can assess robustness.

\section{Proposed Solution Architecture}

The complete solution comprises multiple methodological components (only foundational work completed in this phase):

\textbf{Completed Components:}
\begin{itemize}
    \item Cohort definition using troponin-based clinical adjudication
    \item $\beta$-VAE training for unsupervised ECG representation learning
    \item Comprehensive literature review establishing theoretical foundations
\end{itemize}

\textbf{Future Work:}
\begin{itemize}
    \item Expert-driven DAG construction and structural causal model (SCM) specification
    \item Causal forest implementation for heterogeneous treatment effect (CATE) estimation
    \item Patient-level counterfactual generation using VAE decoder
    \item Rigorous validation using negative controls and E-value sensitivity analysis
\end{itemize}

%===========================================
% CHAPTER 3: LITERATURE SURVEY
%===========================================
\chapter{Literature Survey}

This chapter reviews foundational work across four domains: (1) ECG-based MI detection and benchmarking, (2) causal inference methodologies in healthcare, (3) representation learning with variational autoencoders, and (4) validation and robustness techniques. Fifteen seminal papers are analyzed to establish the state-of-the-art and justify the methodological components of the proposed causal inference pipeline.

\section{ECG-Based MI Detection and Benchmarking}

\subsection{Paper 1: Clinical Trial for Deep Learning MI Detection [1]}

\textbf{Authors:} Mehta et al. (2024)\\
\textbf{Contribution:} ROMIAE multicenter clinical trial demonstrating deep learning model for acute MI detection from 12-lead ECG with sensitivity 91\% and specificity 82\% in real-world clinical settings.\\
\textbf{Methodology:} Deep CNN trained on 1.6 million ECGs, prospectively validated across 10 hospitals, compared against cardiologist interpretation.\\
\textbf{Relevance:} State-of-the-art MI detection benchmark for baseline predictive model comparison; demonstrates clinical feasibility of deep learning ECG analysis.

\subsection{Paper 2: Foundational Theory of Structural Causal Models [2]}

\textbf{Authors:} Pearl (1995)\\
\textbf{Contribution:} Introduced causal diagrams (directed acyclic graphs) as a formal method for representing and analyzing causal relationships in empirical research, establishing the backdoor criterion for identifying causal effects from observational data.\\
\textbf{Methodology:} Graph-theoretic framework defining d-separation, intervention operators (do-calculus), and graphical criteria for causal identification; demonstrates how DAGs encode conditional independence assumptions.\\
\textbf{Relevance:} Foundational theory for DAG design and structural causal model specification; justifies expert-driven causal graph construction using medical domain knowledge to identify confounders and mediators.

\subsection{Paper 3: PTB-XL Benchmark Dataset for ECG Analysis [3]}

\textbf{Authors:} Strodthoff et al. (2020)\\
\textbf{Contribution:} Introduced PTB-XL, the largest publicly available 12-lead ECG dataset with 21,837 records, along with comprehensive benchmarking of deep learning models for multi-label cardiac abnormality classification.\\
\textbf{Methodology:} Evaluated CNNs, LSTMs, and attention mechanisms; macro-averaged AUC of 0.93 for 71 diagnostic classes.\\
\textbf{Relevance:} PTB-XL serves as external validation dataset for ECG feature extractor validation; benchmark results establish performance baselines for predictive modeling.

\section{Causal Inference in Healthcare}

\subsection{Paper 4: Causal Forests for Heterogeneous Treatment Effects [4]}

\textbf{Authors:} Wager \& Athey (2018)\\
\textbf{Contribution:} Introduced honest random forests for estimating conditional average treatment effects (CATE) with theoretical convergence guarantees.\\
\textbf{Methodology:} Splits samples into tree-building and estimation sets, recursive partitioning on covariates, asymptotically normal estimates.\\
\textbf{Relevance:} Provides theoretical foundation for conditional average treatment effect (CATE) estimation to identify heterogeneous patient subgroups.

\subsection{Paper 5: Matching and Propensity Scores [5]}

\textbf{Authors:} Austin (2011)\\
\textbf{Contribution:} Tutorial on propensity score methods for confounding adjustment in observational studies, comparing matching, stratification, and inverse probability weighting.\\
\textbf{Methodology:} Logistic regression for propensity estimation, balance diagnostics using standardized mean differences.\\
\textbf{Relevance:} Baseline comparison method for causal effect estimation; demonstrates traditional approach before modern DML and causal forest techniques.

\subsection{Paper 6: Double Machine Learning for Causal Parameters [6]}

\textbf{Authors:} Chernozhukov et al. (2018)\\
\textbf{Contribution:} Introduced Double Machine Learning (DML) framework for estimating average treatment effects while controlling for high-dimensional confounders.\\
\textbf{Methodology:} Cross-fitting and residualization using any ML model (e.g., Random Forest) to achieve statistically robust, unbiased causal estimates.\\
\textbf{Relevance:} Provides methodology for average treatment effect (ATE) estimation with high-dimensional confounding adjustment in observational data.

\subsection{Paper 7: Expert Knowledge in Causal DAGs [7]}

\textbf{Authors:} Pearce et al. (2023)\\
\textbf{Contribution:} Justifies expert-driven DAG construction in observational health studies as robust alternative to data-driven causal discovery.\\
\textbf{Methodology:} Framework for incorporating domain knowledge into causal graphs, validated on clinical datasets.\\
\textbf{Relevance:} Theoretical basis for our SCM design using medical literature to define confounders.

\section{Representation Learning with VAEs}

\subsection{Paper 8: Causal VAE for Treatment Effects [8]}

\textbf{Authors:} Louizos et al. (2017)\\
\textbf{Contribution:} Introduced Causal Effect Variational Autoencoders (CEVAE) that use latent representations to capture unmeasured confounding in causal inference.\\
\textbf{Methodology:} VAE framework with proxy variables for hidden confounders, demonstrates that learned latent spaces improve CATE estimation.\\
\textbf{Relevance:} Theoretical justification for using $\beta$-VAE latent features to capture unmeasured confounding in causal effect estimation; bridges representation learning with causal inference.

\subsection{Paper 9: $\beta$-VAE for Disentangled Learning [9]}

\textbf{Authors:} Higgins et al. (2017)\\
\textbf{Contribution:} Introduced $\beta$-VAE framework that modifies standard VAE objective to encourage learning of interpretable, disentangled latent factors.\\
\textbf{Methodology:} Weight KL divergence term with $\beta > 1$, demonstrates disentanglement on dSprites and 3D shapes datasets.\\
\textbf{Relevance:} Theoretical basis for our ECG $\beta$-VAE architecture with $\beta = 4.0$.

\subsection{Paper 10: VAE for Medical Time Series [10]}

\textbf{Authors:} Fortuin et al. (2020)\\
\textbf{Contribution:} Demonstrated successful application of Gaussian Process VAE to irregular medical time series, proving VAEs can handle complex temporal healthcare data.\\
\textbf{Methodology:} GP prior in latent space, handles variable-length sequences, evaluated on MIMIC-III and PhysioNet datasets.\\
\textbf{Relevance:} Establishes feasibility of VAE architectures for medical time series analysis, supporting the use of Conv1D VAE for ECG signal representation learning.

\subsection{Paper 11: Interpretability of VAE Latent Space [11]}

\textbf{Authors:} Locatello et al. (2019)\\
\textbf{Contribution:} Large-scale empirical study questioning whether unsupervised disentanglement is achievable without inductive biases.\\
\textbf{Methodology:} Tested 6 VAE variants across 6 datasets with 10,800 model runs, measured 6 disentanglement metrics.\\
\textbf{Insight:} Confirms need for careful validation of disentanglement claims using quantitative metrics.

\section{Validation and Robustness}

\subsection{Paper 12: Sensitivity Analysis with E-Values [12]}

\textbf{Authors:} VanderWeele \& Ding (2017)\\
\textbf{Contribution:} Introduced E-Value metric to quantify robustness of causal estimates to unmeasured confounding.\\
\textbf{Methodology:} Sensitivity parameter that measures minimum strength of confounder-treatment and confounder-outcome associations needed to nullify observed effect.\\
\textbf{Relevance:} Critical validation tool for Phase-2 to assess impact of unobserved variables (e.g., genetics, lifestyle factors).

\subsection{Paper 13: Negative Controls for Confounding Detection [13]}

\textbf{Authors:} Lipsitch et al. (2010)\\
\textbf{Contribution:} Introduced negative control outcomes as a systematic method to detect residual confounding in observational studies.\\
\textbf{Methodology:} Test treatment-outcome associations where no causal effect should exist; if association found, indicates confounding bias.\\
\textbf{Relevance:} Critical methodology for causal validation using negative control outcomes (testing if statin treatment spuriously associates with hospital falls as proof of adequate confounder adjustment).

\section{ECG Counterfactuals and Synthesis}

\subsection{Paper 14: ECG Counterfactual Explanations [14]}

\textbf{Authors:} Sá et al. (2023)\\
\textbf{Contribution:} Demonstrated interpretable MI detection through generative counterfactual ECGs that show minimal changes needed to alter predictions.\\
\textbf{Methodology:} Conditional VAE to generate counterfactual ECG signals, visual explanation interface for clinicians.\\
\textbf{Relevance:} Closest related work; our project extends this from explanation to full causal intervention modeling with patient-level counterfactual generation and treatment effect estimation.

\subsection{Paper 15: Generative Models for Realistic 12-Lead ECG Synthesis [15]}

\textbf{Authors:} Thiam et al. (2023)\\
\textbf{Contribution:} Published in Nature Communications; demonstrated GANs can generate physiologically realistic 12-lead ECG signals indistinguishable from real patient data.\\
\textbf{Methodology:} Wasserstein GAN with gradient penalty, evaluated on clinical plausibility metrics, validated by cardiologists.\\
\textbf{Relevance:} Proves state-of-the-art feasibility of generating realistic counterfactual ECG signals for patient-level interventions; validates that synthetic ECG generation is physiologically plausible.

\section{Summary of Literature}

Table \ref{tab:lit_summary} summarizes the key contributions and pipeline phase mappings of all surveyed papers.

\begin{table}[h]
\centering
\caption{Summary of Literature Survey}
\label{tab:lit_summary}
\begin{tabular}{@{}p{2.5cm}p{3.5cm}p{2.5cm}p{4.5cm}@{}}
\toprule
\textbf{Paper} & \textbf{Method} & \textbf{Dataset} & \textbf{Pipeline Component} \\
\midrule
Mehta et al. & Deep CNN (ROMIAE) & 1.6M ECGs & Baseline predictive models \\
Pearl & SCM, DAG, do-calculus & Theory & DAG design \& SCM \\
Strodthoff et al. & LSTM + Attention & PTB-XL (21K) & Feature validation \\
Wager \& Athey & Causal Forests & CATE estimation \\
Austin & Propensity scores & RHC study & Baseline comparison \\
Chernozhukov et al. & Double ML & Observational & ATE estimation \\
Pearce et al. & Expert DAGs & Clinical studies & Expert-driven DAG \\
Louizos et al. & Causal VAE & IHDP dataset & VAE for confounding \\
Higgins et al. & $\beta$-VAE & dSprites, 3D & VAE architecture \\
Fortuin et al. & GP-VAE & MIMIC-III & Medical time series \\
Locatello et al. & VAE variants & 6 image datasets & Disentanglement validation \\
VanderWeele \& Ding & E-Value & Epidemiology & Sensitivity analysis \\
Lipsitch et al. & Negative controls & Observational & Negative control validation \\
Sá et al. & Counterfactual ECG & PTB-XL & Counterfactual competitor \\
Thiam et al. & ECG-GAN & Clinical data & ECG synthesis \\
\bottomrule
\end{tabular}
\end{table}

%===========================================
% CHAPTER 4: RESEARCH GAPS AND CHALLENGES
%===========================================
\chapter{Research/Technology Gaps and Challenges}

\section{Identified Gaps}

\subsection{Gap 1: Absence of Causal ECG Analysis}

\textbf{Current State:} Existing ECG-based MI detection methods (Papers 1, 3) achieve high predictive accuracy but provide no causal interpretation. Clinicians receive predictions without actionable insights into treatment effects.

\textbf{Evidence:} Even state-of-the-art clinical trials like ROMIAE (Paper 1) focus solely on prediction without estimating patient-specific treatment benefits.

\textbf{Impact:} Limits translation of ML models into clinical decision support tools that recommend personalized interventions.

\subsection{Gap 2: Lack of Disentangled ECG Representations}

\textbf{Current State:} Deep learning models for ECG analysis use entangled representations where latent dimensions capture multiple physiological factors simultaneously, hindering interpretability.

\textbf{Evidence:} Paper 1 (Mehta) and Paper 3 (Strodthoff) employ CNNs and LSTMs without enforcing disentanglement. Paper 9 (Higgins) demonstrates $\beta$-VAE for images but not medical time series.

\textbf{Impact:} Prevents clinicians from understanding which specific ECG features drive model predictions.

\subsection{Gap 3: Integration of Representation Learning with Causal Inference}

\textbf{Current State:} Causal inference methods (Papers 4-7) typically rely on predefined feature sets. Representation learning (Papers 8-11) focuses on reconstruction quality, not causal estimands.

\textbf{Evidence:} Only Paper 8 (Louizos et al.) combines deep learning with causal inference, demonstrating that VAE latent spaces can improve CATE estimation by capturing unmeasured confounding.

\textbf{Impact:} Suboptimal feature representations may introduce bias in CATE estimates or miss critical confounders.

\subsection{Gap 4: Dataset-Specific Challenges in MIMIC-IV}

\textbf{Current State:} MIMIC-IV ECG dataset lacks ground-truth MI labels, requiring clinical adjudication using troponin biomarkers and expert review. Severe class imbalance (12.4\% MI prevalence) complicates model training.

\textbf{Evidence:} Most papers (1, 3, 10, 14) use pre-labeled datasets like PTB-XL or PhysioNet, avoiding the adjudication problem.

\textbf{Impact:} Our work must develop robust troponin-based labeling protocols with clinical validation and address class imbalance through stratified sampling or reweighting.

\section{Technical Challenges}

\subsection{Challenge 1: High-Dimensional ECG Signals}

12-lead ECG recordings with 5000 samples per lead yield 60,000-dimensional input vectors. Convolutional architectures must efficiently extract temporal patterns while avoiding overfitting.

\textbf{Proposed Solution:} 1D convolutional encoder with stride 2 downsampling, reducing dimensionality by factor of 32 (60,000 $\rightarrow$ 1,875) before latent bottleneck.

\subsection{Challenge 2: Disentanglement Validation}

Unsupervised disentanglement is fundamentally unidentifiable without ground-truth factors (Paper 11). We cannot verify if latent dimension $z_5$ corresponds to heart rate without external validation.

\textbf{Proposed Solution:} Quantitative metrics (mutual information, SAP score) and clinical correlation analysis to assess disentanglement quality.

\subsection{Challenge 3: Confounding Bias in Observational Data}

MIMIC-IV is observational, not randomized. Treatment assignment (e.g., statin prescription) depends on patient characteristics, introducing confounding.

\textbf{Proposed Solution:} Directed acyclic graphs (DAG) to identify confounders, propensity score weighting, and sensitivity analysis to assess robustness.

\subsection{Challenge 4: Computational Requirements}

Training $\beta$-VAE on 47,852 ECG records with 82 million parameters requires substantial GPU resources. Single RTX 4050 (6GB VRAM) necessitates batch size optimization and gradient accumulation.

\textbf{Achieved Solution:} Batch size 32, mixed-precision training (FP16), cyclical $\beta$-annealing over 160 epochs converged in 87 epochs with early stopping.

\section{Novelty of Proposed Approach}

Our work addresses the identified gaps through:

\begin{enumerate}
    \item \textbf{First causal ECG framework:} To our knowledge, the first application of CATE estimation to ECG-based MI analysis for personalized treatment recommendations.
    \item \textbf{Disentangled representations:} $\beta$-VAE with rigorous interpretability validation tailored to medical time series.
    \item \textbf{Integrated pipeline:} End-to-end framework combining unsupervised representation learning with causal inference for treatment effect estimation.
    \item \textbf{MIMIC-IV adjudication:} Novel troponin-based labeling protocol for large-scale MI cohort definition with clinical validation.
\end{enumerate}

%===========================================
% CHAPTER 5: DATA EXPLORATION
%===========================================
\chapter{Data Exploration}

\section{Dataset Description}

\subsection{MIMIC-IV Database}

The Medical Information Mart for Intensive Care (MIMIC-IV) v2.2 is a freely accessible critical care database comprising de-identified health records from 299,712 patients admitted to Beth Israel Deaconess Medical Center between 2008-2019. We utilize the following modules:

\begin{itemize}
    \item \textbf{MIMIC-IV-ECG v1.0:} 800,000+ 12-lead ECG recordings (10-second, 500 Hz sampling rate)
    \item \textbf{MIMIC-IV Core:} Demographics (age, sex), admission details, diagnoses (ICD-10 codes)
    \item \textbf{MIMIC-IV Hosp:} Laboratory values (troponin I/T), vital signs, medications
\end{itemize}

\subsection{Cohort Definition}

\subsubsection{Inclusion Criteria}

\begin{enumerate}
    \item Age $\geq$ 18 years at time of ECG
    \item Valid 12-lead ECG recording (all leads present, duration = 10 seconds)
    \item Troponin measurement within 24 hours of ECG
    \item Complete demographic and clinical covariate data
\end{enumerate}

\subsubsection{Exclusion Criteria}

\begin{enumerate}
    \item Pacemaker rhythm (interferes with ST-segment analysis)
    \item Bundle branch blocks (confounds Q-wave interpretation)
    \item Missing or corrupted ECG files
    \item Duplicate records from same patient encounter
\end{enumerate}

\subsubsection{MI Adjudication}

We classify patients into MI vs. Control groups using troponin levels and clinical context:

\textbf{MI Case Definition:}
\begin{itemize}
    \item Troponin I $>$ 0.04 ng/mL OR Troponin T $>$ 14 ng/L (99th percentile thresholds)
    \item \textbf{AND} symptoms of acute coronary syndrome (chest pain, dyspnea) documented in clinical notes
    \item \textbf{AND} ECG changes consistent with ischemia (ST elevation/depression, T-wave inversion)
\end{itemize}

\textbf{Control Definition:}
\begin{itemize}
    \item Symptomatic presentation (chest pain or dyspnea)
    \item Troponin within normal limits
    \item No acute ischemic changes on ECG
\end{itemize}

This yields a labeled cohort of \textbf{47,852 ECG records} (Table \ref{tab:cohort_stats}).

\begin{table}[h]
\centering
\caption{Cohort Statistics}
\label{tab:cohort_stats}
\begin{tabular}{@{}lrr@{}}
\toprule
\textbf{Category} & \textbf{Count} & \textbf{Percentage} \\
\midrule
Total ECG Records & 47,852 & 100.0\% \\
MI Cases & 5,958 & 12.4\% \\
Controls & 41,894 & 87.6\% \\
\midrule
Male & 28,215 & 59.0\% \\
Female & 19,637 & 41.0\% \\
\midrule
Age (mean $\pm$ SD) & \multicolumn{2}{c}{64.3 $\pm$ 15.7 years} \\
Heart Rate (mean $\pm$ SD) & \multicolumn{2}{c}{82.1 $\pm$ 18.4 bpm} \\
\bottomrule
\end{tabular}
\end{table}



%===========================================
% REFERENCES
%===========================================
\begin{thebibliography}{99}

\bibitem{mehta2024}
S. Mehta, M. Lingam, V. Nanda, et al., ``\href{https://academic.oup.com/ehjdh/article/5/2/124/7618032}{Deep learning for the detection of acute myocardial infarction from the 12-lead electrocardiogram: A multi-center validation study (ROMIAE)},'' \textit{European Heart Journal - Digital Health}, vol. 5, no. 2, pp. 124--135, 2024.

\bibitem{pearl1995}
J. Pearl, ``\href{https://doi.org/10.1093/biomet/82.4.669}{Causal diagrams for empirical research},'' \textit{Biometrika}, vol. 82, no. 4, pp. 669--688, 1995.

\bibitem{strodthoff2020}
N. Strodthoff, P. Wagner, T. Schaeffter, and W. Samek, ``\href{https://doi.org/10.1109/JBHI.2020.3022989}{Deep learning for ECG analysis: Benchmarks and insights from PTB-XL},'' \textit{IEEE Journal of Biomedical and Health Informatics}, vol. 25, no. 5, pp. 1519--1528, 2020.

\bibitem{wager2018}
S. Wager and S. Athey, ``\href{https://doi.org/10.1080/01621459.2017.1319839}{Estimation and inference of heterogeneous treatment effects using random forests},'' \textit{Journal of the American Statistical Association}, vol. 113, no. 523, pp. 1228--1242, 2018.

\bibitem{austin2011}
P. C. Austin, ``\href{https://www.ncbi.nlm.nih.gov/pmc/articles/PMC3144483/}{An introduction to propensity score methods for reducing the effects of confounding in observational studies},'' \textit{Multivariate Behavioral Research}, vol. 46, no. 3, pp. 399--424, 2011.

\bibitem{chernozhukov2018}
V. Chernozhukov, D. Chetverikov, M. Demirer, E. Duflo, C. Hansen, W. Newey, and J. Robins, ``\href{https://doi.org/10.1111/ectj.12097}{Double/debiased machine learning for treatment and structural parameters},'' \textit{The Econometrics Journal}, vol. 21, no. 1, pp. C1--C68, 2018.

\bibitem{pearce2023}
N. Pearce, J. A. Lawlor, and S. Vansteelandt, ``\href{https://doi.org/10.1097/EDE.0000000000001605}{Integrating expert knowledge with data in causal models for observational health studies},'' \textit{Epidemiology}, vol. 34, no. 4, pp. 456--465, 2023.

\bibitem{louizos2017}
C. Louizos, U. Shalit, J. M. Mooij, D. Sontag, R. Zemel, and M. Welling, ``\href{https://papers.nips.cc/paper/2017/hash/6446c63156c9c864c264f3c42878b41f-Abstract.html}{Causal effect inference with deep latent-variable models},'' in \textit{Advances in Neural Information Processing Systems (NeurIPS)}, 2017, pp. 6446--6456.

\bibitem{higgins2017}
I. Higgins, L. Matthey, A. Pal, C. Burgess, X. Glorot, M. Botvinick, S. Mohamed, and A. Lerchner, ``\href{https://openreview.net/forum?id=Sy2fzU9gl}{beta-VAE: Learning basic visual concepts with a constrained variational framework},'' in \textit{International Conference on Learning Representations (ICLR)}, 2017.

\bibitem{fortuin2020}
V. Fortuin, D. Baranchuk, G. Rätsch, and S. Mandt, ``\href{http://proceedings.mlr.press/v108/fortuin20a.html}{GP-VAE: Deep probabilistic time series imputation},'' in \textit{International Conference on Artificial Intelligence and Statistics (AISTATS)}, 2020, pp. 1651--1661.

\bibitem{locatello2019}
F. Locatello, S. Bauer, M. Lucic, G. Rätsch, S. Gelly, B. Schölkopf, and O. Bachem, ``\href{http://proceedings.mlr.press/v97/locatello19a.html}{Challenging common assumptions in the unsupervised learning of disentangled representations},'' in \textit{International Conference on Machine Learning (ICML)}, 2019, pp. 4114--4124.

\bibitem{vanderweele2017}
T. J. VanderWeele and P. Ding, ``\href{https://www.acpjournals.org/doi/10.7326/M16-2607}{Sensitivity analysis in observational research: Introducing the E-value},'' \textit{Annals of Internal Medicine}, vol. 167, no. 4, pp. 268--274, 2017.

\bibitem{lipsitch2010}
M. Lipsitch, E. Tchetgen Tchetgen, and T. Cohen, ``\href{https://www.ncbi.nlm.nih.gov/pmc/articles/PMC3053408/}{Negative controls: A tool for detecting confounding and bias in observational studies},'' \textit{Epidemiology}, vol. 21, no. 3, pp. 383--388, 2010.

\bibitem{sa2023}
R. Sá, P. Carvalho, and J. Henriques, ``\href{https://doi.org/10.1016/j.compbiomed.2023.107118}{Interpretable ECG analysis for myocardial infarction detection through counterfactuals},'' \textit{Computers in Biology and Medicine}, vol. 165, pp. 107--118, 2023.

\bibitem{thiam2023}
P. Thiam, P. Bellmann, H. A. Kestler, and G. Palm, ``\href{https://www.nature.com/articles/s41467-023-38840-z}{Generative adversarial networks for realistic 12-lead ECG synthesis},'' \textit{Nature Communications}, vol. 14, pp. 3204, 2023.

\end{thebibliography}

\end{document}
